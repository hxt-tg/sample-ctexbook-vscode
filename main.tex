\documentclass[UTF8,a5paper,zihao=-4,titlepage,oneside]{ctexbook}

\usepackage{hyperref}
\usepackage{titlesec}
\usepackage{amsmath}
\usepackage{amsfonts}
\usepackage{enumitem}
\usepackage{geometry}
\usepackage{import}
\usepackage{graphicx}
% \usepackage{standalone}  % include pdf instead

\usepackage{zhlipsum}
\geometry{left=2.4cm,right=2.4cm,top=3cm,bottom=3cm}

\usepackage{indentfirst}
\setlength{\parindent}{2em}

\titleformat{\chapter}[display]
{\setlength{\parskip}{0pt}\Large\bfseries}{第\,\thechapter\,章}{0pt}{\Huge}
\titlespacing*{\chapter}{20pt}{-35pt}{30pt}
\titleformat{\section}{\Large}{\thesection}{1em}{}
\pagenumbering{Roman}

\begin{document}

\title{
    \zihao{1}测试文本 \\
    测试文本 测试文本}
\author{作者}
\date{本书编译于\today}

\maketitle

\tableofcontents

\frontmatter

\mainmatter

\setlength{\parskip}{0.5em}

\part[测试部分1]
     {测试部分1\\[\bigskipamount] 
      {\normalfont \normalsize 本测试文本测试了测试文本的文本1。}}

\subimport{chap01/}{chap01.tex}
% \chapter{测试章1}

% ============================== [New Section] ==============================
\section{测试节文本1} 

\zhlipsum[1]

测试文本测试文本测试文本测试文本:

\begin{itemize}[label={\checkmark}]
    \item 测试文本测试文本测试文本
    \item 测试文本测试文本测试文本
    \item 测试文本测试文本测试文本
    \item 测试文本测试文本测试文本
\end{itemize}


% ============================== [New Section] ==============================
\section{测试节文本2}

\zhlipsum[2]

\begin{figure}[!htbp]
    \centering
    \includegraphics[width=0.8\textwidth]{figures/sample1.pdf}
    \caption{测试图测试图测试图}
    \label{fig:c01:sample1}
\end{figure}


% ============================== [New Section] ==============================
\section{测试节文本3}


\part[测试部分2]
     {测试部分1\\[\bigskipamount] 
      {\normalfont \normalsize 本测试文本测试了测试文本的文本2。}}

\subimport{chap02/}{chap02.tex}

\setlength{\parskip}{0pt}

\end{document}